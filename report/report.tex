\documentclass{article}

\usepackage{upgreek}

\usepackage{parskip}

\sloppy

\usepackage{amsmath} % actually amsopn
\makeatletter
\DeclareRobustCommand{\var}[1]{\begingroup\newmcodes@\mathit{#1}\endgroup}
\makeatother
\makeatletter
\DeclareRobustCommand{\varb}[1]{\begingroup\newmcodes@\mathbf{#1}\endgroup}
\makeatother

\usepackage{amsfonts}

\usepackage[a4paper, margin=1in]{geometry}

\usepackage{mathtools}
\DeclarePairedDelimiter\ceil{\lceil}{\rceil}
\DeclarePairedDelimiter\floor{\lfloor}{\rfloor}
\DeclarePairedDelimiter\chevrons{\langle}{\rangle}

\usepackage{graphicx}
\graphicspath{{./}}

\usepackage{color}

\definecolor{hyperref}{rgb}{0, 0, 0.4}
\usepackage{hyperref}
\hypersetup{
	colorlinks=true,
	urlcolor=hyperref,
	}

\newcommand{\n}{\ \-}

\usepackage{enumitem}

\usepackage[style=ieee]{biblatex}
\addbibresource{refs.bib}

\usepackage{algorithm}
\usepackage{algpseudocode}

\begin{document}

\section*{Introduction}

This report outlines and analyses an implementation of a parallel algorithm for the following
problem:\\
Given a grid, where each square has a positive integer as its cost, find the shortest 8 connected
path from the top left most square to the bottom right most square, where the distance of a path is
defined as the sum of costs of all squares on the path.

The algorithm I've devised is a modified version of the delta stepping algorithm from Mayer and
Sanders \cite{Meyer-1998}, described in the following pseudocode:
\begin{algorithm}
	\begin{algorithmic}
		\State \(x_{end}\gets x_{size} - 1\)
		\State \(y_{end}\gets y_size - 1\)

		\State \(distance(0, 0)\gets cost(0, 0)\)
	\end{algorithmic}
\end{algorithm}


\printbibliography

\end{document}
